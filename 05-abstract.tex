\begin{abstract}
Deep Learning applied to computer vision has been shown to be able to extract many kinds of semantic information. From classification to localization, or pixel-level semantic segmentation, those new algorithms improved on the state-of-the-art of many tasks and many domains. The company I have been working provides video streaming platforms for many customers. One of them wants to compete with other actors who have been investing in deep learning in order to improve their user experience. We aim to extract semantic information that were not accessible before in order to make better personalized suggestions, emphasize on high quality content and propose new content browsing and exploration features. As such, in this work, we explore tasks such as face identification, activity recognition and recommender systems with an emphasis on latency and the ability to deploy at scale. Our contributions were made made by developing three datasets from our industrial content. The first one is a study on data augmentation and pretrained model to train a classifier from an activity dataset for our data domain. Our second contribution is a survey on learning classifiers in presence of label noise. The next contributions revolve around face recognition. We propose a new loss function, the Threshold-Softmax, aiming to learn from negative samples, that is, faces whose identity is just known not to be one of the other classes. We revert back from metric learning to standard classifiers and explore four loss functions for exploiting further negative learning, using a dataset of faces labeled with their identity, of people famous in our customer's domain. We also contribute a face swapping model based on the VQVAE, along with a proposed new algorithm to improve the vector quantization algorithm. Finally, we use the browsing history of premium users in order to learn a recommender system based on metadata, aiming to mitigate the cold start problem for both users and items.
\end{abstract}

\chapter*{Résumé}
Le Deep Learning appliqué à la vision par ordinateur s'est révélé capable d'extraire de nombreux types d'informations sémantiques. De la classification à la localisation, ou à la segmentation sémantique au niveau du pixel, ces nouveaux algorithmes ont amélioré l'état de l'art de nombreuses tâches et de nombreux domaines. L'entreprise dans laquelle je travaille fournit des plates-formes de streaming vidéo à de nombreux clients. L'un d'entre eux souhaite concurrencer d'autres acteurs qui ont investi dans l'apprentissage profond afin d'améliorer leur expérience utilisateur. Notre objectif est d'extraire des informations sémantiques qui n'étaient pas accessibles auparavant afin de faire de meilleures suggestions personnalisées, de mettre l'accent sur le contenu de haute qualité et de proposer de nouvelles fonctionnalités de navigation et d'exploration du contenu. Ainsi, dans ce travail, nous explorons des tâches telles que l'identification de visage, la reconnaissance d'activité et les systèmes de recommandation en mettant l'accent sur la latence et la capacité de déploiement à grande échelle. Nos contributions ont été réalisées en développant trois jeux de données à partir de notre contenu industriel. La première est une étude sur l'augmentation des données et les modèles pré-entraînés pour entraîner un classificateur à partir d'un ensemble de données d'activité pour notre domaine de données. Notre deuxième contribution est une étude sur l'apprentissage de classifieurs en présence de bruit d'étiquettes. Les contributions suivantes portent sur la reconnaissance des visages. Nous proposons une nouvelle fonction de perte, le Threshold-Softmax, visant à apprendre à partir d'échantillons négatifs, c'est-à-dire des visages dont l'identité n'est pas celle d'une des autres classes. Nous revenons de l'apprentissage métrique aux classificateurs standards et explorons quatre fonctions de perte pour exploiter davantage l'apprentissage négatif, en utilisant un jeu de données de visages étiquetés avec leur identité, de personnes célèbres dans le domaine de notre client. Nous proposons également un modèle d'échange de visages basé sur la VQVAE, ainsi qu'un nouvel algorithme pour améliorer l'algorithme de quantification vectorielle. Enfin, nous utilisons l'historique de navigation des utilisateurs premium afin d'apprendre un système de recommandation basé sur les métadonnées, visant à atténuer le problème du démarrage à froid pour les utilisateurs et les vidéos.

\begin{acknowledgments}

Mes premiers remerciements vont à Hexaglobe, en particulier à Franck COPPOLA et Pierre-Alexandre ENTRAYGUES, et au client que je ne peux nommer, qui m'ont fait confiance et ont parié sur moi. Qui ont laissé place au travail académique et compris les difficultés et incertitudes inhérentes à la recherche, qu'elle soit appliquée ou pas. Ainsi qu'au LIS pour m'avoir accueilli, nommément Elisabeth MURISASCO et Eric BUSVELLE.

J'aimerais tout autant remercier Vincente GUIS pour sa relecture si assidue et dévouée, pour son suivi indéniablement méticuleux, Ricard MARXER et Frédéric BOUCHARA pour l'aide fournie pendant ces années de thèse et avoir accepté cet encadrement et avoir confronté et guidé mes idées.

Merci à la science et tous les humains qui l'ont faite progresser et nous léguer une discipline aussi épanouissante, un monde aussi riche, et une compréhension du monde aussi vaste que nous l'avons maintenant. Merci aux pères fondateurs du Deep Learning pour avoir créé une discipline qui m'a autant intéressé.

Mes remerciements vont également à mes amis et collègues Maxence FERRARI, Marion POUPARD et Paul BEST qui m'ont supporté dans les jours difficiles comme dans les jours de travail jovial, accompagné de ma Air Guitar et de Britney Spears. Qui ont partagé mes réflexions, au développement et à la critique de mes idées, enrichissant grandement ma compréhension, compétence, et mes qualités humaines. (Merci aussi pour le fromage).

Mes pensées chaleureuses vont également à tous les amis qui ont ensoleillé ce parcours. D'abord ceux du Bâtiment X : Baptiste DOMPS, Anatole GROS-MARTIAL, Manon SCHOLIVET, William BRUSCH, Nathan CARRIOT, Alexandre LUTZ, dont certains avec qui j'ai pu partagé des parties de jeu de rôle mémorables. Je n'oublie pas non plus mes amis externes à l'environnement académique : Christophe  et Margaux, Camille et Gautier, les jumeaux ainsi qu'Horgix qui m'accompagne dans mes pérégrinations codistiques depuis plus de 10 ans.

Ma famille également, qui m'a présenté son soutien à bien des moments. Mes petits parents et mon frère, toujours adorables.

Et, plus que tout, exprimer ma reconnaissance envers ma tendre épouse Aurélie qui a été un soutien et un encouragement de chaque jour. Dans les yeux aimants et admiratifs de laquelle j'ai puisé ma force et ma détermination.

\end{acknowledgments}